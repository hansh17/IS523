% SCIS 2018 English manuscript (for LaTeX2e)
\documentclass[a4paper]{article}
\usepackage{Style}
\usepackage{cite}
% If you use LaTeX 2.09, delete the above two lines and remove '%' below.


\begin{document}

\section{Background}

In this section, we are going to briefly explain various materials that we would cover for our attacks. Including variants of the cache attacks we are going to discuss upon attacking the block ciphers, and also some ideas about the searchable encryption.

\subsection{Cache Attacks}

The cache attacks are one of the side channel attacks that is able to recover information about AES T-table or AES key, for even better case. There are 2 most prominent attack, which are called $PRIME+PROBE$ and $FLUSH+RELOAD$.

\subsubsection{PRIME+PROBE}

$PRIME+PROBE$ attacks first loads and occupies a cache set, a set of cache lines, with some dummy code. Then after some moment, measure the time reading data that were loaded on the target cache set. If there was any cache eviction by another process, which in this context the victim, loading would take longer duration as cache miss would occur. This attack doesn't need a shared memory between target and the hostile, but the lowest unit it can attack as a whole set.

\subsubsection{FLUSH+RELOAD}

$FLUSH+RELOAD$ attacks relies upon the inclusive property of the cache policy that is held by the Intel chips. The inclusive property of the cache is that everything on L1 cache should be also on L3 cache. Thus, whenever a cache line is loaded on L1 cache, they are also loaded on L3 cache. Likewise, when a CLFLUSH, or cache flush is called for a cache line within L1 cache, then the same line is evicted from L3 cache.\par
The basic idea of $FLUSH+RELOAD$ is both victim core and the attacker's core load same line of data or instruction from the shared memory. Then the attacker flush the shared cache line from its core. Flushed from the L1 cache, the same line is evicted from L3 cache, and also evicted from the victim's L1 cache as a cascade. Now what we have to observe is that whenever victim cache reloads the same line onto its cache, the very same line is loaded onto the L3 cache again. Meaning that when the attacker reloads the line, the reload time is shortened. This attack obviously need a shared memory between the victim and the attacker, but as an opposite to the $PRIME+PROBE$ attack, it can attack as small as a single line, giving more granuality on the attack.

\subsection{Searchable Encryption}

Used commonly in the email servers, the searchable encryption is an encryption scheme that is designed to facilitate a way to search keyword from an encrypted message without decrpyting it. This scheme is used to encrypt a message from the client and then stored in the external email server where the email service provider is not allowed to read the content of the email for the privacy. When encrypting the message, whether using public key scheme or symmetric scheme, the client also encrypts all the significant words in the message by the words then embed them in the message. Then the server gathers those keywords and index those keywords by the message ids, then store them. Later, when the client queries the keyword what server does is to look up the encrypted query word in the index and return all the messages within that index.
\par However due to the fact that the searching is done through those tags, the hostile, whether it be the server or the eavesdropper, could gather up the tallies on the query as the query keyword is deterministic. Also whenever a result to a query is returned, one, especially the server manager, may easily know which files are accessed. The former is leaking the search pattern and the latter is leaking the access pattern against an account. To prevent malicious attempts to utilize these information leakage there are variants of research on mitigating leakage abuses such as an algorithm to insert paddings without breaking searchability and so. Furthermore, whenever a new file comes into the account, the attacker can easily distinguish whether the message contains certain keyword or not, meaning no forward privacy is held. Fortunately, there are some schemes that protects the forward privacy.

\section{Classification}

\subsection{Symmetric Ciphers and Its Applications}

We now would first introduce several studies about the recent cache related attack, then would also introduce attacks on the searchable encryption. As a notice, these attacks are not developed to solely attack the symmetric ciphers, but are capable in attacking them. For example, the cache attacks doesn't pin point to break the block ciphers, but also is applicable to various privacy breachings such as detecting the mobile touch or swipe, or retrieving RSA keys. Also, the searchable encryption, by itself, can utilize either the symmetic key ciphers or public key ciphers.

\subsubsection{Cache Attacks}

Recent studies on the cache level attacks, such as ARMageddon, are focusing more on either extending pre-existing attack on other platforms such as ARM processors, which may lack some components in launching classic cache attacks, such as $FLUSH+RELOAD$. Other research, $PRIME+ABORT$, focuses on integrating the new hardware features on the cache attacks so that the performance of the attack is improved.

\subsubsection*{ARMageddon}

Our known and famous $FLUSH+RELOAD$ attack was designed to attack the Intel x86 architecture with its cache policy and assembly intact. Meaning that it had the inclusive cache policy with shared cache for all cores in all of its products, an instruction that would easily flush the cache line, and lastly an easy way to time the cache loading. However, ARM core lacks all of those, which in this paper stated as the challenge as ARM architecture has no inclusive cache policy, big.LITTLE makes some of their chips shipped without shared cache, no explicitly known instruction that would flush a cache line flat, and their cache load time is a priviliged instruction.
\par Yet, they successfully tackle these problem by exploiting the cache coherency policy, which loads a cache line from remote CPU if available, and from DRAM if not. This makes them to be able to detect whether the certain line of the cache is loaded on the core other than the attacker's and also independent from the shared cache. Furthermore, they replaced the flushing with a random eviction bringing the idea from the rowhammer attack. Finally, they solved the timing challenge with other instructions such as system call or clock time measuring. This attack made previous $FLUSH+RELOAD$ attack available for ARM architecture, which is widely used in the mobile devices. They were not only able to retrieve the T-Table and the key for AES in openSSL library, but also detect touch or swipe and gather data leaked from ARM TrustZone.

\subsubsection*{PRIME+ABORT}

$PRIME+ABORT$ focuses on integrating new hardware feature for Intel chips, namely Intel TSX. Simply put, Intel TSX is the hardware implementation of transactional memory, which has similar concept to a transaction in the database. Utilizing Intel TSX goes as follows, a transaction of instructions are either completed or aborted in the transactional memory. Among those abortion conditions, there is a rule that aborts a transaction whenever the cache that the transation is currently using is evicted, the transaction is aborted. Using this rule, the attacker would detect the cache eviction right away instead of wating a fixed term and measuing the data read time. Thus, the real attack loads and occupies whole cache set with the transactional memory, which is similar to the $PRIME+PROBE$. Then the attacker waits for the victim to evict the cache. When the victim evicts the cache, then the transaction is aborted right away, letting the attacker know the exact moment the cache is evicted. This attack proves to be more superior to $PRIME+PROBE$ as it doesn't require any kind of timer as eviction is detected right away. Also this attack is much faster and has less noise than the existing one.

\subsubsection{Searchable Encryption Attacks}

The aim for the attackers attacking the searchable encryption is the query recovery, meaning that the attacker needs to find out the plaintext of the encrypted query. The query encryption is significant breach in the privacy as these keywords build up a message which the client saved in the server, and ultimately recover the plaintext for the whole message. The papers included in these attack covers both generic attack and the structural attack on the searchable encryption scheme.

\subsubsection*{All Your Queries are Belong to Us}

This paper delievers a generic attack to recover query's plaintext. Since an attacker can easily insert a malicious message into the victim's messgaes, which is called file injection as the paper says, one can launch a chosen cipher text attack by sending email upon the victim. Now the way to effectively carry out the attack, the paper presents a binary search attack. The idea behind this idea is very similar to the binary search algorithm. We first fix a number of target random keywords $n$, then we now include half of those keywords within the attack message. Then as binary search goes, we select half of keywords that were not selected in the first message and half of those selected in the first message and send the second message consisting of those keywords. If we keep this on, with only file access pattern leakage, we only need to send $log n$ mails to the victim to find out query plaintext when the victim queries server for the list of message by the keyword.
\par However, the above attack procedure requires the message to be long as the attacker blindly guesses the keywords and include them. This problem can be solved when there is a partial leakage of the plaintext to the attacker. (These partial file leakage can be easily achieved as investigating broadcasted emails or announcements) With those given plaintext sets, one can calculate the frequency of occurence of those words within given set and map them to an actual plaintext words. Then when the query kicks in, the attacker calculate the actual frequency of the word, and compare them with pre-calculated frequency list to filter out candidate for possible plaintexts. Using this method, the attacker can narrow down the plaintexts to include in the file injection to carry out binary search attack.

\subsubsection*{The Shadow Nemesis}

Authors of this paper presents a structual attack on the searchable encryption scheme of ShadowCrypt and Mimesis Aegis. These two schemes both have the encrypted keywords tagged in front of the encrypted message, so the attacker can easity figure out which are the encrypted keywords. The difference between those two schemes are that Mimesis Aegis inserts $n$ number of different hashed value for same keywords, the output from Bloom Filters. 
\par The frequency analysis attack, which is the base of this attack, requires a random set of plaintexts. Then the attacker analyze both victim's encrypted keyword frequency and the plaintext's keywords frequency. Compare them, and make guess to those plaintexts based on similarity of those frequencies. This paper uses cocurrency of two words instead of one. Meaning that instead of analyzing single word occurrence, they analyze the frequency of two words occurring at the same time on both victim's keywords and the plaintext's keywords. Then build the graph with each keywords representing the node and the edge representing the co-occurrence frequency to match those words, and the ShadowCrypt is broken with higher success rate than the frequency analysis attack.
\par This doesn't work straight for the Mimesis as they include $n$ values for a single keyword, making the graph's node n times larger without any informative data. Thus, we have to sort those hashed words into single keywords. This can be guessed with frequency analysis upon the number of same hash occurrence. group the hashes with similar frequency, then tally the size of those groups. Most frequent group size is the value $n$ the client's redundant hashes for single keyword. Then we group those hashes together to represent a single keyword and run the same attack we used against the ShadowCrypt.

\section{Evaluation}

\subsection{Symmetric Ciphers and Its Applications}

\begin{table}[]
\centering

\label{my-label}
\begin{tabular}{lll}
                       & P+P & P+A       \\
Shared Memory & No          & No                \\
Granularity            & Cache Set   & Cache Set         \\
Timers & Yes         & No                \\
Dynamic Phase         & No          & Yes               \\
Multiple Sets  & Yes         & Yes(restrictions) \\
Speed                  & Fast        & Faster            \\
Accuracy               & High        & Higher           
\end{tabular}
\caption{Comparison between Prime+Probe(P+P) and Prime+Abort(P+A)}
\end{table}

\subsubsection{Cache Attacks}

We would compare and evaluate each attacks explained to their original motivation. First, we would compare $PRIME+ABORT$ and $PRIME+PROBE$. The major difference from the $PRIME+PROBE$ and the $PRIME+ABORT$ is the requirement of the timer. The former needs an accurate timer while the latter doesn't, making it more responsive on the cache eviction. This makes $PRIME+ABORT$ to have a less room for a noise to kick in, and also doesn't require needless cycle waiting for reload to occur as it is reaped right away. Some disadvantage is that $PRIME+ABORT$ requires more resource, such as threads, to accurately carry out the attack on multiple cache sets while $PRIME+PROBE$ does not. Also $PRIME+ABORT$ requires more recent Intel chips than $PRIME+PROBE$, but since the former can be launched even onto the 4th generation of Intel chips for servers, which makes the attack practical for most cases.
\par For ARMageddon, it is an expanded version of the family of $FLUSH+RELOAD$ attacks. It makes the original attack more feasible as all of cache attacks, $PRIME+PROBE$, $FLUSH+RELOAD$ and even $PRIME+ABORT$ are bound to the Intel architecture. However ARMageddon make the attack possible for ARM architectures. And while doing so, it was able to reduce the requirements for the attack such as flush-free, no shared cache and no inclusive policy on cache management.

\subsubsection{Searchable Encryption Attacks}

The file-injection attack proposed in 'All Your Queries are Belong to Us' is an improved version of file-injection proposed in 2015. The previous file-injection attack assumed the complete file leakage, which made attack less valuable. However, newly presented attack required the partial file leakage or, even better, no file leakage if the message permits. This made the file-injection attacks feasible when the attacker knows less information, making it more meaningful in uncovering cipher texts and practicality. Also this bypasses search query padding mitigation, making the attack harder to defend.
\par The structural attack of Shadow Nemesis is an improved version of the frequency analysis attack. Co-occurrence of two words is tend to give more unique data to compare between the plaintext and the ciphered texts than single word frequencies. Yet, this attack can be partially, if not completely, mitigated if those schemes enforce a dummy keyword inserted in front, rendering more pairs useless as the dummy degrades the uniqueness of word pairs.
\par If we compare two papers introduced, the former is more generic, but it is a chosen plaintext attack. To be more precise, before we filter out and blindly guess the keywords, the attack is a non-adaptive attack. However if we filter out the queries, it makes this attack an adaptive CPA. The latter, meanwhile, cannot be applied to general scheme, but is easy to achieve as it is a known plaintext attack.

\section{Conclusion}

\end{document}
% end of file

